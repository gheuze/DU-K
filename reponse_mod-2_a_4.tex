\documentclass[10pt,english,french]{article}
\usepackage[utf8]{inputenc}
\usepackage[T1]{fontenc}
\usepackage{babel}
\usepackage{lmodern}
\usepackage{hyperref}

% Page length commands go here in the preamble
\setlength{\oddsidemargin}{-0.25in} % Left margin of 1 in + 0 in = 1 in
\setlength{\textwidth}{7in}   % Right margin of 8.5 in - 1 in - 6.5 in = 1 in
\setlength{\topmargin}{-1in}  % Top margin of 2 in -0.75 in = 1 in
\setlength{\textheight}{9.2in}  % Lower margin of 11 in - 9 in - 1 in = 1 in

\renewcommand{\baselinestretch}{1.5} % 1.5 denotes double spacing. Changing it will change the spacing

\babeltags{en=english} % pour utilisation de \texten{...}

\begin{document}
\title{Devoir sur les modules 2 à 4\\
\small{DU épidémiologie des cancers - Isped\\
2018-2019}}
\author{Guillaume Heuzé}
\date{\today}
\maketitle

\section{Premier article\protect\footnote{\bsc{Piel} \emph{et al}. \texten{Increased risk of central nervous system tumours with carbamate insecticide use in the prospective cohort AGRICAN. \emph{Int J Epidemiol} nov 2018}}}

\subsection{Objectif principal}
Bien que les carbamates ne soient pas persistants dans l'environnement, et aient un faible potentiel de bioaccumulation, des études de bio surveillance ont montré leurs présences dans des analyses sanguines ou urinaires après des expositions professionnelles ou, dans une moindre mesure, environnementales. Par ailleurs, des études expérimentales ont montré que les carbamates pouvaient être cancérigènes selon plusieurs mécanismes, dont le stress oxydatif, le cerveau étant considéré comme étant très sensible à celui-ci. Enfin, plusieurs études ont montré un tropisme des carbamates pour le système nerveux central (SNC). L'ensemble de ces éléments a conduit à la définition de l'objectif de cette étude, à savoir étudier l'\textbf{association entre l'incidence des cancers primitifs du SNC (globalement et par sous-type histologique) et l'exposition professionnelle aux insecticides de la classe carbamate}. Cette étude s'inscrit aussi dans le cadre de la demande du Centre international de recherche sur la cancer (Circ), qui a recommandé de réévaluer la cancérogénicité du carbanyl, l'un des pesticides de la classe des carbamates les plus utilisés.

\subsection{Schéma d'étude et justification}
Cette étude est une étude de cohorte prospective pour la partie données de santé. Les avantages de ce type d'étude sont de pouvoir récupérer avec plus de précisions les données de santé et, par ailleurs, de déterminer des risques relatifs comme mesure du risque. Par contre, le principal inconvénient est un coût important, notamment du fait du temps mis pour la récolte des données (ici, le suivi commençant en 2005-2007 - période d'enrôlement - et se finissant fin 2013). 

Par ailleurs, les expositions ont été recueillies de manière rétrospective, à travers l'utilisation des données PESTIMAT (voir \ref{exposition}). L'utilisation, pour l'exposition, de données rétrospectives, permet un gain de temps (les délais entre l'exposition et la survenue des effets sanitaires, pour des substances potentiellement cancérogènes, étant longs) et aussi de pouvoir estimer des expositions à une classe de pesticides dont l'utilisation est en baisse. Cependant, cela ne permet pas une mesure directe de l'exposition. 

\subsection{Population}
La population provient de la cohorte prospective AGRICAN (AGRIculture et CANcer)\footnote{\bsc{Levêque-Morlais} \emph{et al}. \texten{The AGRIculture and CANcer (AGRICAN) cohort study: enrollment and causes of death for the 2005-2009 period. \emph{Int Arch Occup Environ Health} \textbf{2015};88:61-73}}. Celle-ci est composée de 181~842 adultes (hommes ou femmes) habitant dans une zone couverte par un registre des cancers, affiliés à la Mutuelle sociale agricole (MSA) depuis au moins 3~ans à la date du 1\up{er}~janvier~2004. Les personnes pouvaient être actives ou retraitées, propriétaires agricoles, agriculteurs, personnes ayant travaillé en lien avec des activités agricoles (jardiniers, forestiers, travailleurs dans une scierie, etc.). Dans la présente étude, les 10~875~personnes habitant en Côte d'Or n'ont pas été incluses, étant donné que le registre de ce département ne relève pas les cancers du SNC. Par ailleurs, 109~personnes ayant un cancer du SNC prévalent à la date d'inclusion, n'ont pas non plus été incluses. De ce fait, entre 2005 et 2007, 170~858~personnes ont été incluses dans cette étude.

\subsection{Exposition étudiée et façon dont elle a été mesurée}
\label{exposition}
L'exposition étudiée portait sur les carbamates. Elle n'a pas été évaluée directement, mais à partir de l'historique d'activité professionnelle des personnes et de leurs souvenirs d'utilisation de pesticides d'une manière générale. Le questionnaire d'inclusion comportait une partie sur les différentes activités au cours de la vie professionnelle, ainsi que des éléments sur les caractéristiques socio-démographiques, habitudes de vie, données de santé. Un focus était fait sur 13~types de culture et 5~types d'élevages particulièrement suivis dans le cadre de l'étude. Ces données ont été croisées avec les données de la matrice culture - exposition (PESTIMAT), élaborée pour retracer l'utilisation de pesticides en France depuis 1950 dans les principaux secteurs agricoles\footnote{\bsc{Baldi} \emph{et al}. \texten{A French crop-exposure matrix for use in epidemiological studies on pesticides: PESTMAT. \emph{J expo Sci Environ Epidemiol} 2017;\textbf{27}:56-63}}. Pour les utilisations de pesticides en élevage, le dictionnaire des médicaments vétérinaires a été utilisée\footnote{Lepointveterinaire.fr. \emph{\href{https://www.lepointveterinaire.fr/dmv/consulter.html}{Dictionnaire des médicaments vétérinaires et des produits de santé animale commercialisés en France}}}. Cette méthode a permis de s'affranchir d'un biais de mémorisation pour l'utilisation des carbamates.

\subsection{\'{E}vèvement étudié et façon dont les sujets présentant l'évènement ont été identifiés}
L'évènement étudié a été la survenue d'un cancer du SNC. Les données de survie (vivant-décédé, date du décès, au besoin) de l'ensemble des participants, ont été vérifiées annuellement par croisement de la base de l'étude avec les bases de données de la MSA et du répertoire national d'identification des personnes physiques (RNIPP). Les données de cancers ont été vérifiées tous les 2~ans grâce aux données des registres, permettant d'obtenir l'exhaustivité ainsi que les dates précises de diagnostic et le type histologique.

\subsection{Méthode d'analyse et justification}
La méthode d'analyse utilisée est un modèle à risque proportionnel (modèle de Cox). Ce type de modèle permet de prendre en compte des données, même non complètes, ce qui est le cas ici avec un certain nombre de perdus de vue. L'échelle de temps choisie est l'âge à l'inclusion. Pour chaque sous-type histologique, les facteurs de confusion potentiels (sexe, actif/retraité, situation familiale, antécédents tabagiques et alcooliques, existences d'allergies, niveau d'éducation) ont été sélectionnés dès lors qu'ils avaient un p~<~20~\% pour être inclus dans les analyses multivariées. La sélection des variables à prendre finalement en compte a été déterminée par une procédure descendante pas-à-pas.

\subsection{Principaux résultats}
Parmi les 170~858~personnes incluses, 40~964 (24~\%) ont été considérées comme exposées à au moins un carbamate, 57~082 (33,4~\%) à aucun et 72~812 (42,6~\%) avaient un statut d'exposition indéterminé. Elles ont été suivies en moyenne 6,9~années. Durant ce suivi, 381~cas incidents primitifs de cancer du SNC ont été détectés, avec principalement des gliomes (164, représentant 43~\% des cas) et des méningiomes (134, représentant 35~\% des cas). Une augmentation globale du risque relatif de cancers du SNC a été retrouvée chez les utilisateurs de carbamates (RR = 1,47, IC [1,03~; 2,10]). Cette augmentation a aussi été trouvée pour chaque carbamate pris individuellement. Une augmentation linéaire du risque avec la durée d'exposition a été retrouvée pour l'ensemble des cancers du SNC. Malgré le nombre limité de cas exposés, des résultats similaires ont été trouvés par sous-type histologique pour plusieurs molécules. L'analyse par sous-type de cancer a montré des risques relatifs, suivant les différentes molécules, allant de 1,18 à 4,6 pour les gliomes et de 1,51 à 3,67 pour les méningiomes.

\subsection{Forces et limites dont les biais pouvant en découler}
%\subsubsection*{Forces}
Cette étude portait sur une cohorte incluant des personnes ayant exercé dans le domaine de l'agriculture, sans restriction sur le type d'exposition, de sexe, d'âge, etc, permettant ainsi d'avoir un groupe témoin issu de la même population, et donc minimisant un potentiel biais de sélection. Le système de détection des cas, se basant sur les registres, a permis une exhaustivité et une précision des informations. Enfin, la méthode d'évaluation des expositions, a permis d'avoir une bonne évaluation de l'exposition pour une partie de la population.

%\subsubsection*{Limites}
\'{E}tant donné certains oublis d'activités professionnelles, l'exposition n'a pu être estimée pour une part importante de la cohorte (43~\%). Ces personnes étant plus âgés, cela a pu entraîner un biais. Par ailleurs, PESTIMAT ne comprend pas d'informations en terme de probabilités d'utilisation des pesticides, entraînant une surestimation de l'utilisation des insecticides avec une faible probabilité d'utilisation. Cette étude n'a pas pris en compte d'autres facteurs de risques avérés ou suspectés de cancers du SNC. Cependant, ceux-ci n'étant \emph{a priori} pas reliés à l'utilisation des carbamates, cela n'a eu que peu d'influence sur les résultats. Enfin, l'exposition aux autres pesticides n'a pas été prise en compte, ce qui peut avoir un impact sur les résultats. 

\newpage

\section{Deuxième article\protect\footnote{\bsc{Magoni} \emph{et al}. \texten{Plasma levels of polychlorinated biphenyls and risk of cutaneous malignant melanoma: a hospital-based case-control study. \emph{Environ Int.} 2018}}}

\subsection{Objectif principal}
Les principaux facteurs de risques connus du mélanome malin cutané (MMC) sont l'exposition aux rayonnements ultra-violets (UV) d'origine naturelle ou artificielle, les prédispositions génétiques, les caractéristiques phénotypiques, les phototypes, l'âge. Cependant, l'augmentation de l'incidence dans les dernières décennies ne peut être exclusivement expliquée par une augmentation de l'exposition aux UV ou au nombre de coups de soleil. Concernant les facteurs de risque, autre que les UV, le Centre international de recherche sur le cancer (CIRC), a classé les polychloro-biphenyls (PCB) en tant que cancérogène certain pour les MMC.
Cependant, deux méta-analyses récentes n'ont pas trouvé, ou seulement de manière limité, d'association entre l'exposition aux PCB et le risque de MMC et ont conclu qu'il convenait de poursuivre les études. Dans ce cadre, l'objectif de la présente étude était d'\textbf{évaluer l'association entre les concentrations actuelles en PCB dans le sang (globalement et par congénère) et les MMC chez des habitants de la province de Brescia.}

\subsection{Schéma d'étude et justification}
Cette étude est de type cas-témoin. Ce type d'étude a été choisi car il a été fait le choix d'étudier une population particulièrement exposée aux PCB et, de ce fait, en nombre restreint. Dans ce cas, afin d'obtenir un nombre suffisant de personnes, une étude cas-témoin est conseillée. 

\subsection{Population}
La population d'étude a été recrutée entre juillet 2014 et novembre 2016 à l'hôpital \emph{Spedali Civili} de Brescia. Cette province est en effet considérée comme très polluée, à la suite de l'activité d'une entreprise chimique. Les cas, adultes, italiens, de type caucasien, étaient issus du centre de référence des MMC avec confirmation histologique. Les témoins ont été appariés sur l'âge et le sexe et recrutés dans les services de chirurgie et orthopédie du même hôpital, parmi les patients sans histoire de cancers, de maladies hépatiques, endocrines ou auto-immunes. Les données du questionnaire administré concernaient des informations démographiques, de lieu de résidence et de parcours professionnel, les habitudes tabagiques et les expositions aux principaux facteurs de risques connus de MMC.

\subsection{Exposition étudiée et façon dont elle a été mesurée}
Conformément à l'objectif de l'étude, l'exposition étudiée portait sur les PCB et, en l'occurrence, à 33 congénères particuliers. Pour chaque participant à l'étude, 20~ml de sang ont été prélevés pour analyse des PCB, ainsi que des paramètres généraux de biochimie du sang. Les analyses ont été effectuées sans connaître le statut cas-témoin de la personne prélevée. Les niveaux en PCB des échantillons ont été rendus en ng/ml de sang. La valeur totale en PCB a été prise comme la somme des valeurs de chaque PCB pris individuellement. Les PCB étant lipophiles, des valeurs de concentration ajustées sur les lipides ont aussi été rendues et exprimées en ng/g de lipides, ceux-ci étant calculés à partir des valeurs en cholestérol et triglycérides de l'échantillon.

\subsection{\'{E}vèvement étudié et façon dont les sujets présentant l'évènement ont été identifiés}
L'évènement étudié est la survenue (cas incidents) de MMC au sein de personnes venant consultées au sein d'un centre de référence du MMC. Avant inclusion, les cas ont été confirmés histologiquement. Dix-neuf personnes ont rapporté un précédent mélanome, ayant été soigné avec succès et, de ce fait, ont été incluses. \`{A} l'examen histologique, 153~patients (74,6~\%) avaient un mélanome superficiel extensif, 27 (13,2~\%) un mélanome \emph{in situ}, 21 (10,2~\%) un mélanome sur mélanose de Dubreuilh et 4 (2,0~\%) un mélanome nodulaire.

\subsection{Méthode d'analyse et justification}
\'{E}tant donné que nous sommes dans le cadre d'une analyse cas-témoins, la mesure d'association utilisée est l'odds-ratio (OR). Les OR calculés ont été ajustés sur le sexe et l'âge par une régression logistique. Les co-variables finalement retenues ont été déterminées par une procédure descendante pas-à-pas. Les analyses ont été réalisées sur la concentration total en PCB et par congénère, dès lors que ceux-ci étaient détectés chez au moins 50~\% des sujets. Les concentrations en PCB ont été prises en compte de 2 façons : d'une part en tant que variable continue, après transformation logarithmique et, d'autre part, en tant que variable ordinale en utilisant les quantiles de la répartition des concentrations en PCB. Le quartile inférieur a alors été considéré comme la catégorie de référence. Une analyse par sous-groupe a été réalisée chez les personnes plus jeunes d'une part et plus âgées d'autre part, que la médiane (56,67~ans). 

\subsection{Principaux résultats}
Deux cent cinq cas et 205 témoins ont été recrutés. Les associations entre MMC et les caractéristiques personnelles, ainsi que l'exposition aux UV, sont fortement positives. Le niveau d'éducation a aussi été retrouvé associé d'une manière positive au MMC. Une forte association a été retrouvée entre concentrations en PCB et âge. 

Par contre, l'utilisation régulière de moyens de protection solaire n'a pas montré d'association. De même, aucune association n'a été retrouvée avec le fait d'habiter dans une zone avec une forte pollution environnementale. Aucune différence n'a été retrouvée entre les cas et les témoins, que ce soit pour les valeurs de PCB totales ou par congénères. Aucune association n'a non plus été retrouvée en utilisant la concentration en PCB ajustée sur les lipides. Au final, cette étude ne conforte pas l'hypothèse d'une association entre concentration plasmatiques en PCB et survenue de MMC en population générale. \'{E}tant donné qu'aucune association n'a été retrouvée, le risque de MMC suite à une exposition aux PCB doit être inférieur à celui retenu pour le montage cette étude. 

\subsection{Forces et limites dont les biais pouvant en découler}
Cette étude a été menée dans une zone particulièrement polluée, avec une population ayant de ce fait, des concentrations sanguines très élevées. Ce point aurait dû permettre de mieux évaluer le risque de MMC par rapport aux populations non exposée. Les cas et les témoins ont été recrutés dans le même hôpital et provenaient, de ce fait, de la même population, évitant ainsi un biais de recrutement. Le nombre important de personnes a permis d'avoir une puissance assez grande pour détecter un OR statistiquement significatif de 2 au minimum. Les analyses de sang ont été effectuées rapidement, et de manière aveugle.
 
L'étude de l'exposition a consisté en des analyses de sang effectuées sur un prélèvement au moment du diagnostic, alors que les expositions passées auraient été plus pertinentes vis-à-vis d'un risque cancérogène. Cependant, diverses études ont montré le lien entre la concentration en PCB dans le sang et la concentration de ces mêmes PCB dans la graisse sous-cutanée, qui est une mesure acceptable de l'exposition passée. Au final, le nombre limité de personnes habitant dans la zone principalement polluée n'a pas permis d'établir d'association.
\end{document}
