\documentclass[10pt]{article}
\usepackage[utf8]{inputenc}
\usepackage[T1]{fontenc}
\usepackage[frenchb]{babel}
\usepackage{lmodern}
%\usepackage{layout}
%\usepackage[top = 0.3cm, bottom = 1cm]{geometry}

% Page length commands go here in the preamble
\setlength{\oddsidemargin}{-0.25in} % Left margin of 1 in + 0 in = 1 in
\setlength{\textwidth}{7in}   % Right margin of 8.5 in - 1 in - 6.5 in = 1 in
\setlength{\topmargin}{-.75in}  % Top margin of 2 in -0.75 in = 1 in
\setlength{\textheight}{9.2in}  % Lower margin of 11 in - 9 in - 1 in = 1 in

\renewcommand{\baselinestretch}{1.5} % 1.5 denotes double spacing. Changing it will change the spacing

\setlength{\parindent}{0in} 
\begin{document}
\title{Devoir sur les modules 2 à 4\\
%\layout
\small{DU épidémiologie des cancers - Isped\\
2018-2019}}
\author{Guillaume Heuzé}
\date{\today}
\maketitle

\section{Premier article\protect\footnote{Piel et al. Increased risk of central nervous system tumours with carbamate insecticide use in the prospective cohort AGRICAN. Int J Epidemiol 2018}}
\subsection{Objectif principal}
L'objectif principal de cette étude était d'étudier l'association entre l'incidence des cancers du système nerveux central (SNC) sur la période 2005-2013 et l'exposition professionnelle aux insecticides de la classe carbamate, au sein de la cohorte AGRICAN (AGRIculture et CANcer).

\subsection{Schéma d'étude et justification}
Cette étude est une étude de cohorte prospective. Cependant, les expositions ont été recueillies principalementd e manière rétrospective, à travers  l'utilisation d'uen matrice  XXX (PESTIMAT). Les avantages de ce type d'étude sont . Par contre, les inconvénients sont principalement un coût important, notament du fait du temps mis pour la récolte des données.

\subsection{Population}
La population de l'étude est la population (homme ou femme) habitant dans une zone couverte par un registre, affiliée à la Mutuelle sociale agricole (MSA) depuis au moins 3~ans à la date du 1 \up{er} janvier 2004. Les personnes pouvaient être actifs ou retraités, agriculteurs, ouvriers agricoles, personnes aynat travaillé en lien avec des activités agricoles.

\subsection{Exposition étudiée et façon dont elle a été mesurée}

\subsection{\'{E}vèvement étudié et façon dont les sujets présentant l'évènement ont été identifiés}

\subsection{Méthode d'analyse et justification}

\subsection{Principaux résultats}

\subsection{Forces et limites dont les biais pouvant en découler}

\newpage
\section{Deuxième article\protect\footnote{Magoni et al. Plasma levels of polychlorinated biphenyls and risk of cutaneous malignant melanoma: a hospital-based case-control study. Environ Int. 2018}}
\subsection{Objectif principal}

\subsection{Schéma d'étude et justification}

\subsection{Population}

\subsection{Exposition étudiée et façon dont elle a été mesurée}

\subsection{\'{E}vèvement étudié et façon dont les sujets présentant l'évènement ont été identifiés}

\subsection{Méthode d'analyse et justification}

\subsection{Principaux résultats}

\subsection{Forces et limites dont les biais pouvant en découler}

\end{document}
