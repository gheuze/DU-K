\documentclass[10pt]{article}
\usepackage[utf8]{inputenc}
\usepackage[T1]{fontenc}
\usepackage[frenchb]{babel}
\usepackage{lmodern}
%\usepackage[top = 0.3cm, bottom = 1cm]{geometry}

% Page length commands go here in the preamble
\setlength{\oddsidemargin}{-0.25in} % Left margin of 1 in + 0 in = 1 in
\setlength{\textwidth}{7in}   % Right margin of 8.5 in - 1 in - 6.5 in = 1 in
\setlength{\topmargin}{-1in}  % Top margin of 2 in -0.75 in = 1 in
\setlength{\textheight}{9.2in}  % Lower margin of 11 in - 9 in - 1 in = 1 in

\renewcommand{\baselinestretch}{1.5} % 1.5 denotes double spacing. Changing it will change the spacing

\setlength{\parindent}{0in} 
\begin{document}
\title{Devoir sur les modules 2 à 4\\
%\layout
\small{DU épidémiologie des cancers - Isped\\
2018-2019}}
\author{Guillaume Heuzé}
\date{\today}
\maketitle

\section{Premier article\protect\footnote{Piel et al. Increased risk of central nervous system tumours with carbamate insecticide use in the prospective cohort AGRICAN. \emph{Int J Epidemiol} 2018}}
\subsection{Objectif principal}
Bien que les carbonates ne soient pas persistants dans l'environnement, et aient un faible potentiel de bioaccumulation, des études de bio surveillance, menées à travers le monde, ont montré leurs présences dans des analyses sanguines ou urinaires après des expositions professionnelles ou, dans une moindre mesure, environnementales. Par ailleurs, des études expérimentales ont montré que les carbamates pouvaient être cancérigènes selon plusieurs mécanismes, dont le stress oxydatif. Le cerveau est considéré comme étant très sensible à ce type de stress et, par ailleur, plusieurs études ont montré un tropisme des carbamates pour le système nerveux central (SNC). L'ensemble de ces éléments a conduit à la définition de l'objectif de cette étude, à savoir étudier l'\textbf{association entre l'incidence des cancers du SNC (ensemble et par sous-type histologiques) et l'exposition professionnelle aux insecticides de la classe carbamate}. Cette étude s'incrit aussi dans le cadre de la demande du Centre international de recherche sur la cancer (Circ), qui a recommandé de ré-évaluer la cancérogénicité du carbanyl, l'un des pesticides de la classe des carbamates les plus utilisés.

\subsection{Schéma d'étude et justification}
Cette étude est une étude de cohorte prospective pour la partie données de santé. Cependant, les expositions ont été recueillies principalement de manière rétrospective, à travers l'utilisation des données PESTIMAT (voir \ref{exposition}). Les avantages de ce type d'étude (cohorte prospective) sont de \emph{voir cours, pour reprendre ce qu'ils attendent~: pouvoir récupérer plus précisément les données de santé, par exemmple}. Par contre, les inconvénients sont principalement un coût important, notament du fait du temps mis pour la récolte des données (ici, un suivi commençant en 2005-2007 - période d'enrôlement - et se finissant en 2013). L'utilisation de l'utilisation de données rétrospectives permet de gagner du temps et aussi de pouvoir estimer des expositions à une classe de pesticides dont l'utilisation est en baisse. 

\subsection{Population}
La population provient de la cohorte prospective AGRICAN (AGRIculture et CANcer). Celle-ci est composée d'adultes (hommes ou de femmes) habitant dans une zone couverte par un registre des cancers (les personnes habitant en Côte d'Or n'ont pas été incluses, étant donné que le registre de ce départepent ne relève pas les cancers du système nerveux central), affiliés à la Mutuelle sociale agricole (MSA) depuis au moins 3~ans à la date du 1\up{er} janvier 2004. Les personnes pouvaient être actives ou retraitées, agriculteurs, ouvriers agricoles, personnes ayant travaillé en lien avec des activités agricoles (jardiniers, forestiers, travailleurs dans uen scierie, etc.). Au total, entre 2005 et 2007, 181~842 ont été incluses dans cette étude.

\subsection{Exposition étudiée et façon dont elle a été mesurée}
\label{exposition}
L'exposition étudiée portait sur l'exposition aux carbamates (...). Le questionnaire d'inclusion comportait une partie sur l'ensemble des différentes activités au cours de la vie profesionnelle, ainsi que des éléments sur les caractéristiques socio-démographiques, habitudes de vie, données de santé. Le questionnaire demandait aussi si les participant avait travaillé dans des fermes dont l'activité consistaient dans certaines des 13 types de culture ou 5 élevage listés dans le questionnaire. fin du §§. Ces données ont été croisées avec les données de la matrice culture exposition (PESTIMAT), élaborée pour retracer l'utilisation de pesticides en France depuis 1950 dans les principaux secteurs agricoles. Dix-neuf carbamates ont été identifiés.

\subsection{\'{E}vèvement étudié et façon dont les sujets présentant l'évènement ont été identifiés}
L'évènement étudié a été la survenue d'un cancer du SNC. Les données de survie (vivant-décédé, date du décès, au besoin) de l'ensemble des participants, ont été vérifiés annuellement par croisement des bases de données de la MSA et du répertoire national d'identification des personnes physiques (RNIPP). Les données de cancers ont été vérifies tous les 2~ans grace aux données des registres. Grace à ces dernière données, les dates de diagniotic, le type histologique. La classification internationale des maladies pour l'oncologie - CIM-O a été utilisée.

\subsection{Méthode d'analyse et justification}

\subsection{Principaux résultats}

\subsection{Forces et limites dont les biais pouvant en découler}

\newpage
\section{Deuxième article\protect\footnote{Magoni et al. Plasma levels of polychlorinated biphenyls and risk of cutaneous malignant melanoma: a hospital-based case-control study. \emph{Environ Int.} 2018}}
\subsection{Objectif principal}

\subsection{Schéma d'étude et justification}

\subsection{Population}

\subsection{Exposition étudiée et façon dont elle a été mesurée}

\subsection{\'{E}vèvement étudié et façon dont les sujets présentant l'évènement ont été identifiés}

\subsection{Méthode d'analyse et justification}

\subsection{Principaux résultats}

\subsection{Forces et limites dont les biais pouvant en découler}

\end{document}
