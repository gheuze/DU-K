\documentclass[10pt]{article}
\usepackage[utf8]{inputenc}
\usepackage[T1]{fontenc}
\usepackage[frenchb]{babel}
\usepackage{lmodern}
%\usepackage[top = 0.3cm, bottom = 1cm]{geometry}

% Page length commands go here in the preamble
\setlength{\oddsidemargin}{-0.25in} % Left margin of 1 in + 0 in = 1 in
\setlength{\textwidth}{7in}   % Right margin of 8.5 in - 1 in - 6.5 in = 1 in
\setlength{\topmargin}{-1in}  % Top margin of 2 in -0.75 in = 1 in
\setlength{\textheight}{9.2in}  % Lower margin of 11 in - 9 in - 1 in = 1 in

\renewcommand{\baselinestretch}{1.5} % 1.5 denotes double spacing. Changing it will change the spacing

\setlength{\parindent}{0in} 

\begin{document}
\title{Devoir sur les modules 2 à 4\\
\small{DU épidémiologie des cancers - Isped\\
2018-2019}}
\author{Guillaume Heuzé}
\date{\today}
\maketitle

\section{Premier article\protect\footnote{Piel et al. Increased risk of central nervous system tumours with carbamate insecticide use in the prospective cohort AGRICAN. \emph{Int J Epidemiol} 2018}}

\subsection{Objectif principal}
Bien que les carbamates ne soient pas persistants dans l'environnement, et aient un faible potentiel de bioaccumulation, des études de bio surveillance, menées à travers le monde, ont montré leurs présences dans des analyses sanguines ou urinaires après des expositions professionnelles ou, dans une moindre mesure, environnementales. Par ailleurs, des études expérimentales ont montré que les carbamates pouvaient être cancérigènes selon plusieurs mécanismes, dont le stress oxydatif. Le cerveau est considéré comme étant très sensible à ce type de stress et, par ailleur, plusieurs études ont montré un tropisme des carbamates pour le système nerveux central (SNC). L'ensemble de ces éléments a conduit à la définition de l'objectif de cette étude, à savoir étudier l'\textbf{association entre l'incidence des cancers du SNC (ensemble et par sous-type histologiques) et l'exposition professionnelle aux insecticides de la classe carbamate}. Cette étude s'incrit aussi dans le cadre de la demande du Centre international de recherche sur la cancer (Circ), qui a recommandé de ré-évaluer la cancérogénicité du carbanyl, l'un des pesticides de la classe des carbamates les plus utilisés.

\subsection{Schéma d'étude et justification}
Cette étude est une étude de cohorte prospective pour la partie données de santé. Cependant, les expositions ont été recueillies principalement de manière rétrospective, à travers l'utilisation des données PESTIMAT (voir \ref{exposition}). Les avantages de ce type d'étude (cohorte prospective) sont de \emph{voir cours, pour reprendre ce qu'ils attendent~: pouvoir récupérer plus précisément les données de santé, par exemmple}. Par contre, les inconvénients sont principalement un coût important, notamment du fait du temps mis pour la récolte des données (ici, le suivi commençant en 2005-2007 - période d'enrôlement - et se finissant en 2013). L'utilisation, pour l'exposition, de données rétrospectives, permet de gagner du temps (délais entre l'eposition et la survenue des effets sanitaires) et aussi de pouvoir estimer des expositions à une classe de pesticides dont l'utilisation est en baisse. Cependant, cela ne permet pas une mesure directe de l'exposition. 

\subsection{Population}
La population provient de la cohorte prospective AGRICAN (AGRIculture et CANcer). Celle-ci est composée d'adultes (hommes ou de femmes) habitant dans une zone couverte par un registre des cancers (les personnes habitant en Côte d'Or n'ont pas été incluses, étant donné que le registre de ce département ne relève pas les cancers du système nerveux central), affiliés à la Mutuelle sociale agricole (MSA) depuis au moins 3~ans à la date du 1\up{er}~janvier~2004. Les personnes pouvaient être actives ou retraitées, agriculteurs, ouvriers agricoles, personnes ayant travaillé en lien avec des activités agricoles (jardiniers, forestiers, travailleurs dans une scierie, etc.). Au total, entre 2005 et 2007, 181~842 ont été incluses dans cette étude.

\subsection{Exposition étudiée et façon dont elle a été mesurée}
\label{exposition}
L'exposition étudiée portait sur l'exposition aux carbamates (...). Le questionnaire d'inclusion comportait une partie sur l'ensemble des différentes activités au cours de la vie profesionnelle, ainsi que des éléments sur les caractéristiques socio-démographiques, habitudes de vie, données de santé. Le questionnaire demandait aussi si les participants avaient travaillé dans des fermes dont l'activité consistait dans certaines des 13 types de culture ou 5 élevages listés dans le questionnaire. Ces données ont été croisées avec les données de la matrice culture exposition (PESTIMAT), élaborée pour retracer l'utilisation de pesticides en France depuis 1950 dans les principaux secteurs agricoles. Dix-neuf carbamates ont été identifiés.\\
Par ailleurs, dans un contexte professionnel, la voir d'exposition principale est la voie cutanée \emph{(bof, utilité ?)}.

\subsection{\'{E}vèvement étudié et façon dont les sujets présentant l'évènement ont été identifiés}
L'évènement étudié a été la survenue d'un cancer du SNC. Les données de survie (vivant-décédé, date du décès, au besoin) de l'ensemble des participants, ont été vérifiées annuellement par croisement des bases de données de la MSA et du répertoire national d'identification des personnes physiques (RNIPP). Les données de cancers ont été vérifiés tous les 2~ans grâce aux données des registres. Grâce à ces dernière données, les dates de diagnostic, le type histologique. La classification internationale des maladies pour l'oncologie - CIM-O a été utilisée.

\subsection{Méthode d'analyse et justification}
La méthode d'analyse utilisée est un modèle à risque proportionnel. Ce type de modèle permet de prendre en compte des données, même non complètes, ce qui est le cas ici, avec un certain nombre de perdus de vue. L'échelle de temps choisie est l'âge à l'inclusion. Pour chaque sous-type histologique, les facteurs de confusion potentiels (genre, actif/retraité, situation familiale, antécédents tabagiques et alcooliques, existences d'allergies) ont été sélectionnés dès lors qu'ils avaient un p~<~20~\% pour être inclus dans les analyses multivariée. La sélection des variables à prendre finalement en compte a été déterminée par une analyse descendante pas-à-pas.\emph{relire cours sur Cox}

\subsection{Principaux résultats}
Parmi les 170~858~participants, 40~964 (24~\%) ont été considérés comme exposés à au moins un carbamate, 57~082 (33,4~\%) à aucun et 72~812 (42,6~\%) avait un statut d'exposition indéterminé. Ils ont été suivi en moyenne 6,9~années. Durant ce suivi, 381~cas incidents primitifs de cancer du SNC ont été détectés, avec principalement des gliomes (164, représentant 43~\% des cas) et des méningiomes (134, représentant 35~\% des cas). Une augmentation du risque relatif de cancers du SNC a été retrouvée chez les utilisateurs de carbamates (RR = 1,47, IC [1,03~; 2,10]). L'augmentation du risque a été trouvée pour chaque carbamate pris individuellement. Une augmentation linéaire du risque a été retrouvée poru l'ensemble des cancers du SNC avec la durée d'exposition. Cette augmentation a aussi été retrouvé pour les molécules prises individuellement, sauf pour une. Malgré le nombre limité de cas exposés, des résultats similaires ont été trouvés par sous-type histologique pour pkusieurs molécules. L'analyse par sous-type de cancer a montré des facteurs de risque, suivant les différentes molécules, allant de 1,18 à 4,6 pour les gliomes et de 1,51 à 3,67 pour les méningiomes.

\subsection{Forces et limites dont les biais pouvant en découler}
\subsubsection*{Forces}
Cette étude portait sur une cohorte incluant toutes les personnes ayant exercé dans le domaine de l'agriculture, sans restrictions de type d'exposition, de sexe, d'âge, etc, permettant ainsi d'avoir un groupe témoin (personne sans exposition), issu du même groupe de population, et donc minimisant un potentiel biais de sélection. Le système de détection des cas, se basant sur les registres, a permis une exhaustivité et une précision des informations. Enfin, la méthode d'évaluation des expositions, en passant par le parcours professionnel, sans faire appel à la mémoire de l'utilisation de tel ou tel pesticide, a permis d'avoir une bonne évaluation de l'exposition.

\subsubsection*{Limites}
\'{E}tant donné les oublis portant sur certains parcours, l'exposition n'a pû être estimée pour un nombre important de la cohorte (43~\%), ceux-ci étant par ailleurs plus âgés, entrainant de ce fait un possible biais. Par ailleurs, à l'heure actuelle, PESTIMAT ne comprend pas d'informations en terme de probabilités d'utilisation des pesticides, entraînant une surestimation de l'utilisation des insecticides avec des matières actives ayant une faible probabilité d'utilisation. 
Cette étude n'a pas pris en compte d'autres facteurs de risques avérés ou suspectés de cancers du SNC. Cependant, ceux-ci n'étant \emph{a priori} pas reliés à l'utilisation des carbamates, cela n'a eu que peu d'influence sur les résultats. 
Enfin, l'exposition aux autres pesticides n'a pas été prise en compte, ce qui peut avoir un impact sur les résultats. 

\newpage
\section{Deuxième article\protect\footnote{Magoni et al. Plasma levels of polychlorinated biphenyls and risk of cutaneous malignant melanoma: a hospital-based case-control study. \emph{Environ Int.} 2018}}
\subsection{Objectif principal}

\subsection{Schéma d'étude et justification}

\subsection{Population}

\subsection{Exposition étudiée et façon dont elle a été mesurée}

\subsection{\'{E}vèvement étudié et façon dont les sujets présentant l'évènement ont été identifiés}

\subsection{Méthode d'analyse et justification}

\subsection{Principaux résultats}

\subsection{Forces et limites dont les biais pouvant en découler}

\end{document}
