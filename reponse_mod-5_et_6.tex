\documentclass[20pt]{beamer}

\usepackage[utf8]{inputenc}
\usepackage[T1]{fontenc}
\usepackage[french]{babel}
\usepackage{geometry}

\geometry{papersize={33.86cm, 19.05cm}}
%\useoutertheme{shadow}
\usenavigationsymbolstemplate{} % pour supprimer les outils de navigation
\setbeamertemplate{itemize item}[circle]
\definecolor{bleu}{RGB}{47, 85, 151}

\setbeamercolor{structure}{fg=bleu}
 
\setbeamertemplate{footline}{
\leavevmode%
\hbox{\hspace*{-0.06cm}
\begin{beamercolorbox}[wd=.2\paperwidth,ht=2.25ex,dp=1ex,center]{section in head/foot}%
	\usebeamerfont{author in head/foot}\insertshortauthor%~~(\insertshortinstitute)
\end{beamercolorbox}%
\begin{beamercolorbox}[wd=.6\paperwidth,ht=2.25ex,dp=1ex,center]{section in head/foot}%
	\usebeamerfont{section in head/foot}\insertshorttitle
\end{beamercolorbox}%
\begin{beamercolorbox}[wd=.2\paperwidth,ht=2.25ex,dp=1ex,right]{section in head/foot}%
	\usebeamerfont{section in head/foot}\insertshortdate{}\hspace*{2em}
	\insertframenumber{} / \inserttotalframenumber\hspace*{2ex}
\end{beamercolorbox}}%
\vskip0pt%
}




\title[Dépistage du cancer du col de l'utérus]{\textbf{Généralisation du dépistage organisé du cancer du col de l'utérus}}
\subtitle{Plan cancer 2014-2019 - synthèse du cadre et recommandations}

\author[G. Heuzé]{Guillaume Heuzé}

\institute{Devoir 2 DU épidémiologie des cancers, Université de Bordeaux, ISPED}

\date{\today}


\begin{document}
	\frame{
		\titlepage
		}

	\frame{
		\frametitle{Plan de la présentation}		
		\tableofcontents
		}
\section{Contexte}
	\frame{
		\frametitle{Histoire naturelle de la maladie}
		\begin{itemize}
			\item virus responsable \textit{exclusif ?} du cancer : papillomavirus humain (\textit{human papillomavirus} - HPV)~;
			\item infection lors du début de la vie sexuelle (70~\% des femmes vont être en rapport avec un HPV, dont 40~\% lors de premiers rapports~;
			\item régression spontanée de l'infection chez 85~\% des jeunes filles après 24 mois~;
			\item virus peuvent être à l'origine de lésions~:
				\begin{itemize}
					\item de bas grade, qui se développent en 2 à 5~ans et disparaissent spontanément pour 99~\%d'entre eux~,
					\item de haut grade, qui se développent en 6 à 8~ans, dont seulement 5 à 15~\% évolueront vers un cancer invasif
				\end{itemize}
			\item persistance de l'infection chez les autres jeunes filles, avec potentiellement des lésions précancéreuses, évolution possible, après 15 à 20~ans, en cancer~;
		\end{itemize}
	}
	\frame{
		\frametitle{Contexte épidémiologique}
		\begin{itemize}
			\item environ 3~000~ nouveaux cas par an en france métropolitaine en 2012 ;
			\item 11\ieme cancer féminin (2~\% des cancers de la femme)~;
			\item âge médian au diagnostic en 2012 : 51 ans (53 en 2005)~; 
			\item environ 1~100~décès par an estimé en 2012 ;
			\item âge médian au décès en 2012 : 64 ans ;
			\item survie nette (survie qui serait observée si la seule cause de mortalité chez les femmes atteintes était le cancer du col de l'utérus) à 5~ans de 66~\% et, à 10~ans, de 59~\% ;
			\item \textit{données de prévalence ?}.
		\end{itemize}
		}
\section{\'{E}tat des lieux}
	\frame{
		\frametitle{\'{E}tat des lieux du dépistage à l'heure actuelle en France}
		\begin{itemize}
			\item l'Inca recommande à l'heure actuelle le dépistage du cancer du col de l'utérus chez les femmes de 25 à 65~ans, par un frottis cervical utérien (FCU), tous les 3 ans, après 2 frottis normaux à un an d'intervalle(dépistage individuel)~;
			\item sur la période 2004-2006, le nombre de frottis réalisés correpondrait à un taux de couverture de 89,4~\% si les femmes de faisaient qu'un seul dépistage tous les 3~ans. Ce chiffre cache 2 réalités~:
				\begin{itemize}
					\item certaines femmes se font surdépistées,
					\item à l'inverse, d'autres se font sous-dépistées, ou, non dépistées, en particulier des femmes en situation socio-économique fragiles. Cela entraîne des diagnostics tardifs,
				\end{itemize}
			\item ce taux est en réalité d'environ 61~\%. Il a permis une diminution de l'incidence du cancer du col de l'utérus de 2,5~\% par an entre 1980 et 2012~;
			\item celà entraîne \textit{conséquences des sur et sous-dépistages}
				\begin{itemize}
					\item cs des sur-diagnostics,
					\item les sous-diagnostics entraînent des retards au diagnostic et donc des pronostics aggravés et des survies diminuées,
				\end{itemize}
		\end{itemize}
		Ces éléments ont amenés la Haute Autorité de santé (HAS) à recommander la mise en place d'un programme organisé. Celui-ci a été inscrit dans le plan cancer 2014-2019\footnote{action 1.1~: permettre à chaque femme de 25 à 65~ans l'accès à un dépistage régulier du cancer du col utérin \textit{via} un programme national de dépistage organisé}. L'objectif est un taux de couverture de dépistage de 80~\% dans la population cible, en particulier dans les populations les plus vulnérables.
		}
  	\frame{
		\frametitle{Retour d'expériences des dépistages mis en place dans les autres pays}
		En France, 13~départements ont mis en place un dépistage organisé, à titre expérimental, depuis XXX, résultats
		Dans les autres pays européen, annexe5 GRED


		Mais sinon, démarche individuelle ne permet pas à toutes les femems d'accéder à ce dépistage, en aprticulier, les femmes en situation de précarité.
		par ailleurs, surcoût (page 12), intervalle entre 2 tests non respecté, etc..
		}
	\frame{
		\frametitle{Diapositive type}
		p. 6. Un dispositif de dépistage nécessite plusieurs pré-requis :
		\begin{itemize}
			\item concerner une pathologie grave et d'incidence élevée ;\textcolor{bleu}{essai}
			\item reposer sur un ou plusieurs tests sensibles (déf), spécifiques(déf), peu couteux et d'application aisée pour être acceptables par la population ciblée ;
			\item être fondé sur un système qualité et d'évaluation de la démarche ;
			\item atteindre un taux de participation suffisamment élevé afin d'être efficient.
			\item \textit{notion de durée de développement du cancer}

		\end{itemize}
		}

	\frame{
		\frametitle{Utilité du dépistage alors même de l'existence d'un vaccin}
		Il existe une méthode efficace de prévention primaire par la vaccination. Cette vaccination ne dispense cependant pas du déopistage pour plusieurs raisons~:
		\begin{itemize}
			\item l'ensemble de la population cible actuelle du dépistage n'a pas été vacciné (et cette vaccination, pour être efficace, doit être effectuée avant l'infection, c'est à dire avant le début de la vie sexuelle)~;
			\item le taux de couverture de la vaccination reste faible actuellement (de l'ordre de 18~\% en 2015)~;
			\item les 2 vaccins autorisés protègent de 2 papillomavirus différents, responsables d'un nombre important de cancers, mais d'autres génotypes, non inclus dans les vaccins, sont aussi oncogènes\footnote{\textit{note aux correcteurs : je me suis placé dans le contexte des documents. \'{A} l'heure actuele, le vaccin préconisé pour tout début de vaccination, comprend 9 génotypes}}
		\end{itemize}
		Environ X\% des cancers du col de l'utérus ont pour origine une infection par le papillomavirus humain. Cependant, dans le vaccin, pas l'ensemble des sérotypes, plus CV débute, et faible.
		}

	\frame{
		\frametitle{quelle population, quels actions}
		notion d'universalisme proportionnée :
		programme préconisé par l'Inca (page 18) vise à mettre en place des stratégies ciblées en fonction des catégories de femmes identifiées.
		pas de discrimination positive car cela entraîne de la stigmatisation, et, à l'inverse, un sentiment d'injustice pout ceux qui n'en bénéficient pas.
		Cependant, droit des personnes (p. 19), le choix de faire, ou pas, le dépistage, après un choix éclairé
		}	
    
    	\frame{
		\frametitle{Techniques de dépistage}
		Il existe à l'heure actuelle 3 techniques de dépistage~:
		\begin{itemize}
			\item \textbf{frottis cervico-utérien(FCU)}~: prélèvement de cellule au niveau du col utérin, suivi d'une analyse cytologique, permettant d'identifier des cellules précancéreuses~;
			\item \textbf{test HPV (recherche du génome des virus)}~: prélèvement au niveau du col utérin, puis détection par recherche de l'ADN du virus~;
			\item \textbf{double marquage immunologique p16/Ki67}~: technique de marquage immunologique des cellules dont la combinaison permet d'estimer le risque évolutis des lésions précancéreuse.
		\end{itemize}
	}

\section{\'{E}valuation médico-économique}
	\frame{
		\frametitle{\'{E}valuation médico-économique}
		\begin{itemize}
			\item une évaluation, réalisée par l'Inca, a montré que la mise en place du DO (invitation des femmes et relance des femmes non-spontanément participantes) permettrait une amélioration par rapport à la situation actuelle (dépistage spontané), en termes de cancers évités, de survie et de survie ajustée par la qualité de vie des femmes.
			\item l'évaluation montre que la stratégie de référence, \textbf{d'un point de vue médico-économique} serait un test HPV tous les 10~ans. Cette stratégie est cependant associée à une réduction moindre de l'incidence par rapport à un DO fondé sur le FCU tous les 3~ans, rendant cette solution incompatible avec le plan cancer~;
			\item les deux statégies, l'une basée sur un FCU tous les 3~ans avec envoi du kit d'auto-prélèvement HPV à la relance et l'autre, fondé sur un test HPV tous les 5 ans sont les stratégies le splus efficientes~;
			\item cette 2\ieme stratégie est celle préconisée. Cependant, les conditions actuelles de sa mise en oeuvre ne sont pas remplies.
		\end{itemize}
		}
\section{Préconisations de l'Inca pour la généralisation du DO}
	\frame{
		\frametitle{Préconisations de l'Inca pour la généralisation du DO}
		\'{E}tant donné les contraintes de calendrier, des risques identifiés et des prérequis, non satisfaits à ce jour, pour le passage au test HPV, l'Inca préconise que le DO se mette en place~:
		\begin{itemize}
			\item en créant les conditions du passage à terme au test HPV en dépistage primaire~;
			\item en tenant compte à court terme de la hiérarchisation des stratégies de DO fondée sur le FCU~;
			\item en mettant en place les évaluations nécessaires
		\end{itemize}
		}

    
    
\section{recommandations}    
    	\frame{
		\begin{center}
		\bsc{\textcolor{bleu}{\Huge{\textbf{Recommandations}}}}
		\end{center}
		}
\end{document}
