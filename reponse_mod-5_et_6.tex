\documentclass[20pt]{beamer}

\usepackage[utf8]{inputenc}
\usepackage[T1]{fontenc}
\usepackage[french]{babel}
\usepackage{geometry}

\geometry{papersize={33.86cm, 19.05cm}}
%\useoutertheme{shadow}
\usenavigationsymbolstemplate{} % pour supprimer les outils de navigation
\setbeamertemplate{itemize item}[circle]
\definecolor{bleu}{RGB}{47, 85, 151}

\setbeamercolor{structure}{fg=bleu}
 
\setbeamertemplate{footline}{
\leavevmode%
\hbox{\hspace*{-0.06cm}
\begin{beamercolorbox}[wd=.2\paperwidth,ht=2.25ex,dp=1ex,center]{section in head/foot}%
	\usebeamerfont{author in head/foot}\insertshortauthor%~~(\insertshortinstitute)
\end{beamercolorbox}%
\begin{beamercolorbox}[wd=.6\paperwidth,ht=2.25ex,dp=1ex,center]{section in head/foot}%
	\usebeamerfont{section in head/foot}\insertshorttitle
\end{beamercolorbox}%
\begin{beamercolorbox}[wd=.2\paperwidth,ht=2.25ex,dp=1ex,right]{section in head/foot}%
	\usebeamerfont{section in head/foot}\insertshortdate{}\hspace*{2em}
	\insertframenumber{} / \inserttotalframenumber\hspace*{2ex}
\end{beamercolorbox}}%
\vskip0pt%
}




\title[Dépistage du cancer du col de l'utérus - synthèse et recommandations]{\huge{\textbf{Généralisation du dépistage organisé du cancer du col de l'utérus}}}
\subtitle{Synthèse et recommandations dans le cadre du plan cancer 2014-2019}

\author[G. Heuzé]{Guillaume Heuzé}

\institute{Devoir 2 - DU épidémiologie des cancers - Université de Bordeaux, ISPED}

\date{\today}


\begin{document}
	\frame{
		\titlepage
		}

	\frame{
		\frametitle{Plan de la présentation}
		\tableofcontents
		}

\section{Synthèse des éléments sur le dépistage du cancer du col de l'utérus} 
\subsection{Contextes médical, épidémiologique, institutionnel et international}
	\frame{
		\frametitle{Contexte médical}
		\begin{itemize}
		\item virus responsable du cancer du col de l'utérus~: \textbf{papillomavirus humain (\textit{human papillomavirus} - HPV)}. Existence de plus de 150~sérogroupes, dont une 15\up{aine} identifiés comme oncogènes~;
		\item infection lors du début de la vie sexuelle (\textbf{70~\% des femmes vont être en rapport avec un HPV, dont 40~\% lors des premiers rapports})~;
		\item régression spontanée de l'infection chez 85~\% des jeunes femmes après 24 mois~;
		\item certains sérogroupes peuvent être à l'origine de lésions, après une persistance de l'infection~:
			\begin{itemize}
			\item de bas grade, qui se développent en 2 à 5~ans et disparaissent spontanément pour 99~\% d'entre eux~;
			\item de haut grade, qui se développent en 6 à 8~ans, dont seulement 5 à 15~\% évolueront vers un cancer invasif, 15 à 20~ans après l'infection.
			\end{itemize}
		\end{itemize}
	}
	
	\frame{
		\frametitle{Contexte épidémiologique}
		\begin{itemize}
		\item \textbf{environ 3~000~nouveaux cas par an en France métropolitaine en 2012}~;
		\item 11\ieme~cancer féminin (2~\% des cancers de la femme)~;
		\item âge médian au diagnostic en 2012 : 51 ans (53 en 2005)~; 
		\item \textbf{environ 1~100~décès par an estimés en 2012}~;
		\item âge médian au décès en 2012 : 64 ans~;
		\item \textbf{survie nette} (survie qui serait observée si la seule cause de mortalité chez les femmes atteintes était le cancer du col de l'utérus) \textbf{à 5~ans de 66~\% et, à 10~ans, de 59~\%}.
		\end{itemize}
		}

  	\frame{
		\frametitle{Contexte institutionnel et international}
		L'Organisation mondiale de la santé (OMS) demande à ce que ...

		\bigskip
		En 2008, la Commission européenne a mis en place un cadre commun formalisé et l'a mis à jour en 2015.

		\bigskip
		Le dépistage a été mis en place dans de nombreux pays européens. L'analyse de 15 d'entre eux, montre qu'il n'y a pas de corrélation entre le type de dépistage existant (démarche individuelle ou programme organisé) et l'importance de la réduction de l'incidence.

		\bigskip
		Cette corrélation existe avec l'ancienneté du dépistage. Les taux d'incidence les plus bas (<7/100~000) sont en général les pays ayant commencé avant ou dans les années 1970-80.
		En France, la mise en place du DO a été inclus dans la Plan cancer 2014-2019.
		}

\subsection{Utilité du dépistage dans le contexte d'existence d'un vaccin}
	\frame{
		\frametitle{Utilité du dépistage dans le contexte d'existence d'un vaccin}
		La \textbf{vaccination} contre le HPV constitue une méthode efficace de prévention primaire. 
		Cette vaccination ne dispense cependant pas du dépistage pour plusieurs raisons~:
		\begin{itemize}
		\item \textbf{l'ensemble de la population cible} actuelle du dépistage \textbf{n'a pas été vacciné} (et cette vaccination, pour être efficace, doit être effectuée avant l'infection, c'est à dire avant le début de la vie sexuelle)~;
		\item le \textbf{taux de couverture} de la vaccination \textbf{reste faible} actuellement (de l'ordre de 18~\% en 2015)~;
		\item les \textbf{2 vaccins autorisés} protègent de 2 papillomavirus différents, responsables d'un nombre important de cancers, mais d'\textbf{autres génotypes, non inclus dans les vaccins, sont aussi oncogènes}\footnote{\textit{note aux correcteurs : je me suis placé dans le contexte des documents. \'{A} l'heure actuelle, le vaccin préconisé pour tout début de vaccination, comprend 9~génotypes.}}.
		\end{itemize}
		\bigskip
		\large{=> en parallèle de la mise en place de la vaccination, le dépistage s'avère donc nécessaire.}
		}

\subsection{Recommandations actuelles}
    	\frame{
		\frametitle{Recommandations actuelles}
		L'Inca recommande le dépistage du cancer du col de l'utérus chez les femmes de 25 à 65~ans, tous les 3 ans, après 2~dépistages normaux à un an d'intervalle. En l'absence d'un système de DO actuellement, ce dépistage est réalisé de manière individuelle (ou spontané).

		\bigskip
		La technique de dépistage utilisée est le frottis cervico-utérien (FCU)~: prélèvement de cellule au niveau du col utérin, suivi d'une analyse cytologique conventionnelle, permettant d'identifier des cellules pré-cancéreuses.
		
		\bigskip
		Par ailleurs, une des autres possibilités de dépistage est le test HPV (recherche du génome des virus)~: prélèvement au niveau du col utérin, puis détection par recherche de l'ADN du virus. Nous y reviendrons plus loin.
	}

	\frame{
		\frametitle{Principes d'un dépistage}
		Un dispositif de dépistage ne peut être mis en place pour toute maladie. Certaines conditions sont nécesssaires afin qu'il soit efficace~: % p.6 pour les pré-requis et 36 pour la mise en place
		\begin{itemize}
		\item concerner une pathologie grave et d'incidence élevée~: 
		\item reposer sur un ou plusieurs tests sensibles (proportion de personnes réellement malades identifié par un test), spécifiques (proportion de personnes réellement non malades, identifié comme non malade par un test), peu coûteux et d'application aisée pour être acceptables par la population ciblée ;
		\item être fondé sur un système qualité et d'évaluation de la démarche ;
		\item avoir une évolution lente entre le moment où les cellules pré-cancéreuses sont retrouvées et leur transformation en cancer, permettant un traitement.
		\end{itemize}
		
		\bigskip
		Le cancer du col de l'utérus répond à ces différents critères.
		}
		
\subsection{Pourquoi mettre en place un dépistage organisé}
  	\frame{
		\frametitle{Pourquoi mettre en place un dépistage organisé}
		Sur la période 2004-2006, le nombre de frottis réalisés correspondrait à un taux de couverture de plus de 89~\% si les femmes ne faisaient qu'un seul dépistage tous les 3~ans. Ce chiffre masque d'importantes disparités~:
		\begin{itemize}
		\item chez certaines femmes, principalement en situation socio-économique aisée ou relativement jeunes~: début du dépistage trop précoce, non respect des intervalles entre 2~dépistages~;
		\item chez les autres, en situation socio-économique plus fragile, ou plus âgées~: sous-dépistage ou non-dépistage, entraînant des retards au diagnostic et donc des pronostics aggravés et des survies diminuées.
		\end{itemize}

		\bigskip
		De ce fait, ce taux de couverture est en réalité d'environ 55 à 60~\%, avec 40 à 50~\% des femmes concernées qui ne participent pas, ou de façon irrégulière. % p.48
		
		\bigskip
		Même s'il a permis une diminution de l'incidence du cancer du col de l'utérus de 2,5~\% par an entre 1980 et 2012, le dépistage individuel est donc moins efficace que le dépistage organisé, en termes de réduction de l'incidence et de la mortalité. Ces éléments ont amenés la Haute Autorité de santé (HAS) à recommander la mise en place d'un programme organisé, au travers de l'inscription de celui-ci dans le plan cancer 2014-2019\footnote{action 1.1~: permettre à chaque femme de 25 à 65~ans l'accès à un dépistage régulier du cancer du col utérin \textit{via} un programme national de dépistage organisé}, comme précédemment mentionné. % p. 49
		Le cancer du col de l'utérus répond à ces différents critères mais ne permet pas d'atteindre un taux de participation suffisamment élevé afin d'être efficient~;
		
		\bigskip	
		Une évaluation médico-économique, réalisée par l'Inca, a montré que la \textbf{mise en place du DO} permettrait une \textbf{amélioration par rapport à la situation actuelle}, en termes de cancers évités, de survie et de survie ajustée par la qualité de vie des femmes. Les réductions attendues d'incidence et de mortalité sont comprises entre 13~\% et 26~\%. Les gains d'espérance de vie atteignent 35 à plus de 60~ans pour 10~000~femmes. 
		}


\subsection{\'{E}lements sur la mise en place d'un dépistage}

	\frame{
		\frametitle{Réflexions préalables à la mise en place d'un dépistage}
		Si la maladie répond aux conditions nécessaires, sa mise en place nécessite plusieurs étapes de réflexion~:
		\begin{itemize}
		\item identification de la population cible~;
		\item modalités d'invitation des femmes éligibles à ce dépistage~;
		\item recueil des frottis~;
		\item lecture des frottis et recueil des résultats~;
		\item information des femmes ayant un résultat normal et de l'intervalle à respecter avant le prochain frottis~;
		\item relance des femmes ayant un frottis non satisfaisant ou ininterprétable~;
		\item suivi des femmes ayant un frottis anormal~;
		\item enregistrement, suivi et évaluation du programme.
		\end{itemize}
		
		\bigskip
		L'expérimentation d'un dépistage organisé dans 13~départements a permis de préciser les différentes conditions de mise en \oe{}uvre. Cette mise en place a entraîné une augmentation de 5 à 15 points du taux de couverture. 
		}
	

\subsection{Quelle population, quelles actions}
	\frame{
		\frametitle{Quelle population, quelles actions}
		Le passage à un dépistage national implique la prise en compte des inégalités sociales de santé, l'ampleur et l'intensité des actions devant être adaptées aux situations des différents groupes (notion d'universalisme proportionné)~:
		\begin{itemize}
			\item population vulnérable ou éloignées du système de santé (prostituées, Roms, migrantes, \textit{etc.})~;
			\item femmes avec risque aggravé de cancer du col (VIH, immunodépression, Distilbène)~;
			\item femmes non sensibilisées (couverture maladie universelle complémentaire, affection de longue durée, homosexuelles)~;
			\item femmes de plus de 50~ans.
		\end{itemize}

		\bigskip
		Ce choix a été fait, plutôt que la mise en place d'une discrimination positive, car cette dernière peut entraîner des effets contre productifs~:
		\begin{itemize}
			\item stigmatisation des personnes invitées au dépistage (puisqu'appartenant à des groupes à risque)~;
			\item sentiment d'injustice de la part des femmes ne bénéficiant pas des invitations.
		\end{itemize}

		\bigskip
		La prise en compte se fera par relance des femmes non-spontanément répondantes.
		}	
    

\subsection{Quelle technique de dépistage mettre en place, et à quelle fréquence}
	\frame{
		\frametitle{Quelle technique de dépistage mettre en place, et à quelle fréquence}
		
		Dans le cadre de son étude médico-économique, l'Inca a évalué plusieurs stratégies en termes de technique de dépistage et d'intervalle de dépistage~:
		\begin{itemize}
		\item cette évaluation montre que la stratégie de référence, \underline{d'un point de vue médico-économique}, serait un test HPV tous les 10~ans. Cette stratégie est cependant associée à une réduction moindre de l'incidence par rapport à un DO fondé sur le FCU tous les 3~ans, rendant cette solution incompatible avec le plan cancer~;
		\item les \textbf{deux stratégies les plus efficientes} sont~:
			\begin{itemize}
			\item un FCU tous les 3~ans avec envoi du kit d'auto-prélèvement HPV à la relance,
			\item un test HPV tous les 5 ans,
			\end{itemize}
		\item cette \textbf{2\ieme~stratégie est celle préconisée}. Cependant, \textbf{les conditions actuelles de sa mise en \oe{}uvre ne sont pas remplies}.\textit{pourquoi ?}.
		\end{itemize}
		
		\bigskip
		De ce fait, l'Inca préconise que le DO se mette en place~:
		\begin{itemize}
		\item en créant les conditions du passage à terme au test HPV tous les 5~ans en dépistage primaire~;
		\item en tenant compte à court terme de la hiérarchisation des stratégies de DO fondée sur le FCU~;
		\item en mettant en place les évaluations nécessaires. \textit{p. 58 s'il faut développer}
		\end{itemize}
		}

\subsection{Conclusions globales}
	\frame{
		\frametitle{Conclusions globales}
		\begin{itemize}
		\item le cancer du col de l'utérus n'est pas le plus fréquent chez la femme (11\ieme~cancer féminin en terme d'incidence, représentant \og seulement\fg~2~\% des cancers)~;
		\item cependant, la survie nette est de 2~femmes sur 3, au bout de 5~ans, et a par ailleurs diminué entre 1990 (68~\%) et 2002 (64~\%)~;
		\item les techniques de dépistage et de traitement existent et sont performantes~;
		\item le dépistage, tel qu'il est effectué à l'heure actuelle, c'est à dire individuellement, entraîne une non-égalité entre les femmes, notamment celles ayant un niveau socio-économique plus faible.
		\end{itemize}
		
		L'objectif est un taux de couverture de dépistage de 80~\% dans la population cible, en particulier dans les populations les plus vulnérables. 
		=> importante de la mise en \oe{}uvre du dépistage organisé et ce, dans le cadre d'un universalisme proportionné, c'est à dire en renforcement l'action envers les femmes les plus fragiles.
		}    
    
\section{Recommandations}    
    	\frame{
		\frametitle{Recommandations}
		Dans le cadre de la mise en place du dépistage organisé, plusieurs points de vigilance sont à surveiller~:
		\begin{itemize}
			\item respect de la liberté individuelle~:
				\begin{itemize}
					\item autonomie et déontologie médicale,
					\item équité et bienfaisance,
				\end{itemize}
			\item justification du choix des actions ciblées~:
				\begin{itemize}
					\item justice,
					\item responsabilité,
				\end{itemize}
			\item éviter le risque de stigmatisation dans les interventions ciblées~:
				\begin{itemize}
					\item non-malfaisance,
				\end{itemize}
			\item assurer l'accès aux soins~:
				\begin{itemize}
					\item équité,
					\item non malfaisance,
					\item justice,
				\end{itemize}
			\item évaluer et anticiper les évolutions~:
				\begin{itemize}
					\item bienfaisance
					\item non-malfaisance,
					\item justice,
					\item confiance.
				\end{itemize}
		\end{itemize}
		%\begin{center}
		%\bsc{\textcolor{bleu}{\Huge{\textbf{Recommandations}}}}
		%\end{center}
		}
\end{document}
