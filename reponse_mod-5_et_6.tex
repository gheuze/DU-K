\documentclass[20pt]{beamer}

\usepackage[utf8]{inputenc}
\usepackage[T1]{fontenc}
\usepackage[french]{babel}
\usepackage{geometry}

\geometry{papersize={33.86cm, 19.05cm}}
%\useoutertheme{shadow}
\usenavigationsymbolstemplate{} % pour supprimer les outils de navigation
\setbeamertemplate{itemize item}[circle]
\definecolor{bleu}{RGB}{47, 85, 151}
%\addtobeamertemplate{footline}{\hfill\insertframenumber/\inserttotalframenumber}

\setbeamercolor{structure}{fg=bleu}
 
\setbeamertemplate{footline}{
\leavevmode%
\hbox{\hspace*{-0.06cm}
\begin{beamercolorbox}[wd=.2\paperwidth,ht=2.25ex,dp=1ex,center]{author in head/foot}%
	\usebeamerfont{author in head/foot}\insertshortauthor%~~(\insertshortinstitute)
\end{beamercolorbox}%
\begin{beamercolorbox}[wd=.6\paperwidth,ht=2.25ex,dp=1ex,center]{section in head/foot}%
	\usebeamerfont{section in head/foot}\insertshorttitle
\end{beamercolorbox}%
\begin{beamercolorbox}[wd=.2\paperwidth,ht=2.25ex,dp=1ex,right]{section in head/foot}%
	\usebeamerfont{section in head/foot}\insertshortdate{}\hspace*{2em}
	\insertframenumber{} / \inserttotalframenumber\hspace*{2ex}
\end{beamercolorbox}}%
\vskip0pt%
}




\title[dépistage du cancer du col de l'utérus]{\textbf{Généralisation du dépistage du cancer du col de l'utérus}}
\subtitle{synthèse du cadre et recommandations}

\author{Guillaume Heuzé}

\institute{Devoir 2 DU épidémiologie des cancers, Université de Bordeaux, ISPED}

\date{\today}


\begin{document}
\section{intro}
	\frame{
		\titlepage
		}

	\frame{
		\frametitle{Plan de la présentation}		
		\tableofcontents
		}
\section{Contexte}
	\frame{
		\frametitle{Contexte épidémiologique}
		\begin{itemize}
			\item environ 3~000~ nouveaux cas par an en france métropolitaine en 2012 ;
			\item environ 1~100~décès par an estimé en 2012 ;
			\item survie nette (pour les femmes seulement DCD de ça) à 5~ans de 66~\% et, à 10~ans, de 59~\% ;
		\end{itemize}
		}
	\frame{
		\frametitle{Diapositive type}
		p. 6. Un dispositif de dépistage nécessite plusieurs pré-requis :
		\begin{itemize}
			\item concerner une pathologie grave et d'incidence élevée ;\textcolor{bleu}{essai}
			\item reposer sur un ou plusieurs tests sensibles (déf), spécifiques(déf), peu couteux et d'application aisée pour être acceptables par la population ciblée ;
			\item être fondé sur un système qualité et d'évaluation de la démarche ;
			\item atteindre un taux de participation suffisamment élevé afin d'être efficient.
			\item \textit{notion de durée de développement du cancer}

		\end{itemize}
		pourquoi un dépistage, alors qu'il existe le vaccin
		}


	\frame{
		\frametitle{Utilité du dépistage alors même de l'existence d'un vaccin}
		Environ X\% des cancers du col de l'utérus ont pour origine une infection par le papillomavirus humain. Cependant, dans le vaccin, pas l'ensemble des sérotypes, plus CV débute, et faible.
		}

  	\frame{
		\frametitle{Retour d'expériences des dépistages mis en place dans les autres pays}
		balbla.

		Mais sinon, démarche individuelle ne permet pas à toutes les femems d'accéder à ce dépistage, en aprticulier, les femmes en situation de précarité.
		par ailleurs, surcoût (page 12), intervalle entre 2 tests non respecté, etc..
		}
	\frame{
		\frametitle{quelle population, quels actions}
		notion d'universalisme proportionnée :
		programme préconisé par l'Inca (page 18) vise à mettre en place des stratégies ciblées en fonction des catégories de femmes identifiées.
		pas de discrimination positive car cela entraîne de la stigmatisation, et, à l'inverse, un sentiment d'injustice pout ceux qui n'en bénéficient pas.
		Cependant, droit des personnes (p. 19), le choix de faire, ou pas, le dépistage, après un choix éclairé
		}	
    
    	\frame{
		\frametitle{Conclusion globale}
	}
    
    
\section{recommandations}    
    	\frame{
		\begin{center}
		\Large{\textbf{Recommandations}}
		\end{center}
		}
\end{document}
