\documentclass[20pt]{beamer}

\usepackage[utf8]{inputenc}
\usepackage[T1]{fontenc}
\usepackage[french]{babel}
\usepackage{geometry}

\geometry{papersize={33.86cm, 19.05cm}}
%\useoutertheme{shadow}
\usenavigationsymbolstemplate{} % pour supprimer les outils de navigation
\setbeamertemplate{itemize item}[circle]
\definecolor{bleu}{RGB}{47, 85, 151}

\setbeamercolor{structure}{fg=bleu}
 
\setbeamertemplate{footline}{
\leavevmode%
\hbox{\hspace*{-0.06cm}
\begin{beamercolorbox}[wd=.2\paperwidth,ht=2.25ex,dp=1ex,center]{section in head/foot}%
	\usebeamerfont{author in head/foot}\insertshortauthor%~~(\insertshortinstitute)
\end{beamercolorbox}%
\begin{beamercolorbox}[wd=.6\paperwidth,ht=2.25ex,dp=1ex,center]{section in head/foot}%
	\usebeamerfont{section in head/foot}\insertshorttitle
\end{beamercolorbox}%
\begin{beamercolorbox}[wd=.2\paperwidth,ht=2.25ex,dp=1ex,right]{section in head/foot}%
	\usebeamerfont{section in head/foot}\insertshortdate{}\hspace*{2em}
	\insertframenumber{} / \inserttotalframenumber\hspace*{2ex}
\end{beamercolorbox}}%
\vskip0pt%
}




\title[Dépistage du cancer du col de l'utérus - synthèse et recommandations]{\huge{\textbf{Généralisation du dépistage organisé du cancer du col de l'utérus}}}
\subtitle{Synthèse et recommandations dans le cadre du plan cancer 2014-2019}

\author[G. Heuzé]{Guillaume Heuzé}

\institute{Devoir 2 - DU épidémiologie des cancers - Université de Bordeaux, ISPED}

\date{\today}


\begin{document}
	\frame{
		\titlepage
		}

	\frame{
		\frametitle{Plan de la présentation}
		\tableofcontents
		}

\section{Synthèse des éléments sur le dépistage du cancer du col de l'utérus} 
\subsection{Contextes médical, épidémiologique, institutionnel et international}
	\frame{
		\frametitle{Contexte médical}
		\begin{itemize}
		\item virus responsable du cancer du col de l'utérus (CCU)~: \textbf{papillomavirus humain (\textit{human papillomavirus} - HPV)}. Existence de plus de 150~sérogroupes, dont une 15\up{aine} identifiés comme oncogènes~;
		\item infection lors du début de la vie sexuelle (\textbf{70~\% des femmes vont être en rapport avec un HPV, dont 40~\% lors des premiers rapports})~;
		\item régression spontanée de l'infection chez 85~\% des jeunes femmes après 24 mois~;
		\item certains sérogroupes peuvent être à l'origine de lésions, après une persistance de l'infection~:
			\begin{itemize}
			\item de bas grade, qui se développent en 2 à 5~ans et disparaissent spontanément pour 99~\% d'entre eux,
			\item de haut grade, qui se développent en 6 à 8~ans, dont seulement 5 à 15~\% évolueront vers un cancer invasif, 15 à 20~ans après l'infection.
			\end{itemize}
		\end{itemize}
	}
	
	\frame{
		\frametitle{Contexte épidémiologique}
		\begin{itemize}
		\item \textbf{environ 3~000~nouveaux cas par an en France métropolitaine en 2012}~;
		\item 11\ieme~cancer féminin (2~\% des cancers de la femme)~;
		\item âge médian au diagnostic en 2012 : 51 ans (53 en 2005)~; 
		\item \textbf{environ 1~100~décès par an estimés en 2012}~;
		\item âge médian au décès en 2012 : 64 ans~;
		\item \textbf{survie nette} (survie qui serait observée si la seule cause de mortalité chez les femmes atteintes était le cancer du col de l'utérus) \textbf{à 5~ans de 66~\% et, à 10~ans, de 59~\%}.
		\end{itemize}
		}

\subsection{Recommandations actuelles}
    	\frame{
		\frametitle{Recommandations actuelles}
		L'Inca recommande le dépistage du CCU chez les \textbf{femmes de 25 à 65~ans, tous les 3 ans, après 2~dépistages normaux à un an d'intervalle}. En l'absence d'un système de DO actuellement, ce \textbf{dépistage} est réalisé de manière \textbf{individuelle} (ou spontané).

		\bigskip
		La technique de dépistage utilisée est le \textbf{frottis cervico-utérien (FCU)}~: prélèvement de cellule au niveau du col utérin, suivi d'une analyse cytologique conventionnelle, permettant d'identifier des cellules pré-cancéreuses. Ce dépistage a montré son \textbf{efficacité} dans la réduction de l'incidence et de la mortalité.
		
		\bigskip
		Par ailleurs, une des autres possibilités de dépistage est le \textbf{test HPV} (recherche du génome des virus)~: prélèvement au niveau du col utérin, puis détection par recherche de l'ADN du virus. Nous y reviendrons plus loin.
	}

	\frame{
		\frametitle{Principes d'un dépistage}
		Le dépistage des cancers consiste à identifier, au sein d'\textbf{une population non demandeuse de soins, les personnes asymptomatiques} atteintes d'un cancer ou d'une lésion précancéreuse. De ce fait, la responsabilité des pouvoirs publics est engagée et le programme se doit de \textbf{maximiser les bénéfices et minimiser les effets délétères}.

		\bigskip
		Un dispositif de dépistage ne peut être mis en place pour toute pathologie. Certaines conditions sont nécesssaires afin qu'il soit efficace~: % p.6 pour les pré-requis et 36 pour la mise en place
		\begin{itemize}
		\item concerner une \textbf{pathologie grave et d'incidence élevée}~; 
		\item reposer sur un ou plusieurs \textbf{tests sensibles} (proportion de personnes réellement malades identifié par un test), \textbf{spécifiques} (proportion de personnes réellement non malades, identifié comme non malade par un test), \textbf{peu coûteux et d'application aisée} pour être acceptables par la population ciblée ;
		\item être fondé sur un \textbf{système qualité et d'évaluation de la démarche} ;
		\item avoir une \textbf{évolution lente} entre le moment où les cellules pré-cancéreuses sont retrouvées et leur transformation en cancer, permettant un traitement.
		\end{itemize}
		
		\bigskip
		Le CCU répond à ces différents critères.
		}

  	\frame{
		\frametitle{Contexte institutionnel}
		En 2008, un rapport de la Commission européenne souligne que l'\textbf{efficacité d'un dépistage spontané reste limité} et que celui-ci ne doit pas être encouragé.
		
		\bigskip	
		En 2010, la Haute Autorité de santé (HAS) a recommandé la mise en place d'un \textbf{programme de dépistage organisé (DO)}. 

		\bigskip
		La 1\iere{} étape a été une expérimentation de la mise en place de ce dépistage dans 13 départements, qui a entraîné une augmentation de 5 à 15 points du taux de couverture.
		
		\bigskip
		La recommandation de la HAS a été reprise dans les \textbf{objectifs du Plan cancer 2014-2019} qui mentionne la généralisation du dépistage du CCU\footnote{action 1.1~: permettre à chaque femme de 25 à 65~ans l'accès à un dépistage régulier du cancer du col utérin \textit{via} un programme national de dépistage organisé}, avec, comme objectif la réduction de l'incidence et de la mortalité de ce cancer. Pour y arriver, les objectifs opérationnels sont~:
		\begin{itemize}
		\item un taux de couverture de 80~\%~;
		\item rendre ce dépistage plus accessible aux populations les plus vulnérables ou les plus éloignés du système de santé (recommandations de l'OMS et de la Commission européenne).
		\end{itemize}
	}	

\subsection{Utilité du dépistage dans le contexte d'existence d'un vaccin}
	\frame{
		\frametitle{Utilité du dépistage dans le contexte d'existence d'un vaccin}
		La \textbf{vaccination} contre le HPV constitue une méthode efficace de prévention primaire. 
		La mise en place de cette vaccination ne dispense cependant pas de l'extension du dépistage pour plusieurs raisons~:
		\begin{itemize}
		\item \textbf{l'ensemble de la population cible} actuelle du dépistage \textbf{n'a pas été vacciné} (et cette vaccination, pour être efficace, doit être effectuée avant l'infection, c'est à dire avant le début de la vie sexuelle)~;
		\item le \textbf{taux de couverture} de la vaccination \textbf{reste faible} actuellement (de l'ordre de 18~\% en 2015)~;
		\item les \textbf{2 vaccins autorisés} protègent de 2 papillomavirus différents, responsables d'un nombre important de cancers, mais d'\textbf{autres génotypes, non inclus dans les vaccins, sont aussi oncogènes}\footnote{\textit{note aux correcteurs : je me suis placé dans le contexte des documents. \'{A} l'heure actuelle, le vaccin préconisé pour tout début de vaccination, comprend 9~génotypes.}}.
		\end{itemize}
		\bigskip
		\large{=> en parallèle de la mise en place de la vaccination, le dépistage s'avère donc nécessaire.}
		}
	
	\frame{
		\frametitle{titre à trouver}
		Dans le cadre du Plan cancer et de l'extension du DO, 2 rapports ont été produits~:
		\begin{itemize}
		\item \textbf{évaluation médico-économique du dépistage du cancer du col de l'utérus en France}, par l'Institut national du cancer (Inca). L'objectif était d'évaluer l'efficience des différentes modalités de DO~;
		\item \textbf{généralisation du dépistage organisé du CCU~: quel cadre éthique ?} préconisations du groupe de réflexion sur l'éthique du dépistage (Gred).
		\end{itemize}

		\bigskip
		Ces 2~rapports permettent de poser les bases de la généralisation du DO et notamment de répondre aux questions sur les techniques de dépistage et les intervalles de dépistage.
		}
	
	\frame{
		\frametitle{Retour sur le contexte européen}
		Le dépistage a été mis en place dans de nombreux pays européens. L'analyse de 15 d'entre eux, montre qu'il n'y a pas de corrélation entre le type de dépistage existant (démarche individuelle ou programme organisé) et l'importance de la réduction de l'incidence.

		\bigskip
		Cette corrélation existe avec l'ancienneté du dépistage. Les taux d'incidence les plus bas (<7/100~000) sont en général les pays ayant commencé avant ou dans les années 1970-80.
		}



		
\subsection{Pourquoi mettre en place un dépistage organisé}
  	\frame{
		\frametitle{Pourquoi mettre en place un dépistage organisé}
		Sur la période 2004-2006, le nombre de frottis réalisés correspondrait à un taux de couverture de plus de 89~\% si les femmes ne faisaient qu'un seul dépistage tous les 3~ans. Ce chiffre masque d'\textbf{importantes disparités}~:
		\begin{itemize}
		\item chez \textbf{certaines femmes}, principalement en situation socio-économique aisée ou relativement jeunes~: début du dépistage trop précoce, non respect des intervalles entre 2~dépistages, ce qui entraîne un \textbf{sur-dépistage}~;
		\item chez les \textbf{autres}, en situation socio-économique plus fragile, ou plus âgées~: \textbf{sous-dépistage ou non-dépistage}, entraînant des retards au diagnostic et donc des pronostics aggravés et des survies diminuées.
		\end{itemize}

		\bigskip
		De ce fait, ce \textbf{taux de couverture} est en réalité d'environ \textbf{55 à 60~\%}, avec \textbf{40 à 50~\% des femmes concernées qui ne participent pas, ou de façon irrégulière}. % p.48
		
		\bigskip
		Même s'il a permis une diminution de l'incidence du cancer du col de l'utérus de 2,5~\% par an entre 1980 et 2012, le \textbf{dépistage individuel est donc moins efficace que le DO}, en termes de réduction de l'incidence et de la mortalité. 	
		}

	\frame{
		\frametitle{Conclusions sur l'évaluation médico-économique}	
		\bigskip	
		
		L'évaluation médico-économique, réalisée par l'Inca, a été conduite avec une perspective collective tous payeurs (\'{E}tat, Assurance maladie, assurances complémentaires de santé, femmes).

		\bigskip
		Cette évaluation a montré que la \textbf{mise en place du DO} permettrait~:
		\begin{itemize}
		\item  une \textbf{amélioration par rapport à la situation actuelle}, en termes de cancers évités, de survie et de survie ajustée par la qualité de vie des femmes~;
		\item les \textbf{réductions} attendues d'\textbf{incidence et de mortalité} sont comprises entre \textbf{13~\% et 26~\%}~;
		\item les \textbf{gains d'espérance de vie atteignent 35 à plus de 60~ans pour 10~000~femmes}.
		\end{itemize}
		}



\subsection{Quelle technique de dépistage mettre en place, et à quelle fréquence}
	\frame{
		\frametitle{Quelle technique de dépistage mettre en place, et à quelle fréquence}
		
		Toutes les stratégies de DO par invitation et relance des femmes non spontanément participantes permettent d'améliorer la couverture du dépistage et de diminuer l'incidence et la mortalité liée au CCU~:
		\begin{itemize}
		\item la stratégie de référence, \underline{d'un point de vue médico-économique}, serait un test HPV tous les 10~ans. Cette stratégie est cependant associée à une réduction moindre de l'incidence par rapport à un DO fondé sur le FCU tous les 3~ans, rendant cette solution incompatible avec le plan cancer~;
		\item les \textbf{deux stratégies les plus efficientes} sont~:
			\begin{itemize}
			\item un FCU tous les 3~ans avec envoi du kit d'auto-prélèvement HPV à la relance,
			\item un test HPV tous les 5 ans,
			\end{itemize}
		\item cette \textbf{2\ieme~stratégie est celle préconisée}. Cependant, \textbf{les conditions actuelles de sa mise en \oe{}uvre ne sont pas remplies}.
		\end{itemize}
		
		\bigskip
		De ce fait, l'Inca préconise que le DO se mette en place~:
		\begin{itemize}
		\item en créant les conditions du passage à terme au test HPV tous les 5~ans en dépistage primaire~;
		\item en tenant compte à court terme de la hiérarchisation des stratégies de DO fondée sur le FCU~;
		\item en mettant en place les évaluations nécessaires. 
		\end{itemize}
		}

\subsection{Quelle population, quelles actions}
	\frame{
		\frametitle{Quelle population, quelles actions}
		Le passage à un \textbf{dépistage national} implique la \textbf{prise en compte des inégalités sociales de santé}, l'ampleur et l'intensité des actions devant être adaptées aux situations des différents groupes (notion d'\textbf{universalisme proportionné})~:
		\begin{itemize}
			\item population vulnérable ou éloignées du système de santé (prostituées, Roms, migrantes, \textit{etc.})~;
			\item femmes avec risque aggravé de cancer du col (VIH, immunodépression, Distilbène)~;
			\item femmes non sensibilisées (couverture maladie universelle complémentaire, affection de longue durée, homosexuelles)~;
			\item femmes de plus de 50~ans.
		\end{itemize}

		\bigskip
		Ce choix a été fait, plutôt que la mise en place d'une \textbf{discrimination positive}, car cette dernière peut entraîner des \textbf{effets contre productifs}~:
		\begin{itemize}
			\item stigmatisation des personnes invitées au dépistage (puisqu'appartenant à des groupes à risque)~;
			\item sentiment d'injustice de la part des femmes ne bénéficiant pas des invitations.
		\end{itemize}

		\bigskip
		Les impacts budgétaires résultants des actions ciblées sont très faibles au regard du coût total de la généralisation du DO.
		}	
    

\subsection{Conclusions globales}
	\frame{
		\frametitle{Conclusions globales}
		Suite aux rapports, il convient de déployer en France un programme de DO du CCU par FCU triennal puis, à terme, par test HPV tous les 5~ans. 

		\bigskip
		En effet, bien que le cancer du col de l'utérus ne soit pas le plus fréquent chez la femme (11\ieme~cancer féminin en terme d'incidence, représentant \og seulement\fg~2~\% des cancers)~:
		\begin{itemize}
		\item la \textbf{survie nette} est de 2~femmes sur 3, au bout de 5~ans, et a par ailleurs \textbf{diminué} entre 1990 (68~\%) et 2002 (64~\%)~;
		\item les \textbf{techniques de dépistage et de traitement existent et sont performantes}~;
		\item le \textbf{dépistage}, tel qu'il est effectué à l'heure actuelle, c'est à dire \textbf{individuellement}, entraîne une \textbf{non-égalité entre les femmes}, notamment celles ayant un \textbf{niveau socio-économique plus faible}.
		\end{itemize}
		
		=> importante de la mise en \oe{}uvre du dépistage organisé et ce, dans le cadre d'un universalisme proportionné, c'est à dire en renforcement l'action envers les femmes les plus fragiles.
		}    
    
\section{Recommandations}    

	\frame{
		\begin{center}
		\bsc{\textcolor{bleu}{\Huge{\textbf{Recommandations}}}}
		\end{center}
		}


    	\frame{
		\frametitle{Les points de vigilance}
		Dans le cadre de la mise en place du dépistage organisé, plusieurs points de vigilance sont à surveiller~:
		\begin{itemize}
			\item respecter la liberté individuelle, en particulier assurer l'exercice de l'autonomie des femmes dans la décision en s'appuyant sur des outils d'information et d'accompagnement adaptées~:
			\item éviter les risques de stigmatisation dans les interventions ciblées afin de minimiser le sentiment de culpabilisation en étant vigilant sur la façon de communiquer~;
			\item justifier le choix des actions ciblées et leur priorité en tenant compte des résultats des expérimentions. Ce processus décisionnel requiert une transparence dans les critères qui guideront les choix et l'allocation des ressources disponibles~:
			\item organiser le parcours des femmes afin de garantir l'accès aux soins et limiter les risques de parcours incomplets et les pertes de chances~:
			\item évaluer le dispositif pour anticiper les évolutions et assurer l'adaptabilité du programme. Ceci est d'autant plus important qu'une montée en charge progressive du dispositif, fondée sur des évaluation intermédiaires, est prévue.
		\end{itemize}
		}
	\frame{
		\frametitle{titre à trouver}
	Pour développer les différents outils de communication, il convient de s'appuyer sur~:
	\begin{itemize}
	\item des groupes représentatifs des différentes communautés lorsqu'ils existent (Roms, homosexuelles, \textit{etc.})~;
	\item les bases de données médico-administratives (CMUc, ALD, \textit{etc.}).
	\end{itemize}
	}
\end{document}
